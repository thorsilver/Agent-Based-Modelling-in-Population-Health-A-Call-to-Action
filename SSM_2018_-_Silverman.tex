\documentclass[review]{elsarticle}

\usepackage{lineno,hyperref,array,pbox}
\modulolinenumbers[5]

\journal{Social Science and Medicine}

%%%%%%%%%%%%%%%%%%%%%%%
%% Elsevier bibliography styles
%%%%%%%%%%%%%%%%%%%%%%%
%% To change the style, put a % in front of the second line of the current style and
%% remove the % from the second line of the style you would like to use.
%%%%%%%%%%%%%%%%%%%%%%%

%% Numbered
%\bibliographystyle{model1-num-names}

%% Numbered without titles
%\bibliographystyle{model1a-num-names}

%% Harvard
\bibliographystyle{model2-names.bst}\biboptions{authoryear}

%% Vancouver numbered
%\usepackage{numcompress}\bibliographystyle{model3-num-names}

%% Vancouver name/year
%\usepackage{numcompress}\bibliographystyle{model4-names}\biboptions{authoryear}

%% APA style
%\bibliographystyle{model5-names}\biboptions{authoryear}

%% AMA style
%\usepackage{numcompress}\bibliographystyle{model6-num-names}

%% `Elsevier LaTeX' style
%\bibliographystyle{elsarticle-num}
%%%%%%%%%%%%%%%%%%%%%%%

\begin{document}

\begin{frontmatter}

\title{Agent-Based Modelling in Population Health:\\ A Call to Action}
%%\tnotetext[mytitlenote]{Fully documented templates are available in the elsarticle package on \href{http://www.ctan.org/tex-archive/macros/latex/contrib/elsarticle}{CTAN}.}

%% Group authors per affiliation:
\author[mymainaddress]{Eric Silverman\corref{mycorrespondingauthor}}
\cortext[mycorrespondingauthor]{Corresponding author}
\ead{eric.silverman@glasgow.ac.uk}
\author[mymainaddress]{Mark McCann}
\author[mymainaddress]{Laurence Moore}
\author[mymainaddress]{Umberto Gostoli}
\author[mysecondaryaddress]{Richard Shaw}
\author[mytertiaryaddress]{Claudio Angione}


%%\address{MRC/CSO Social and Public Health Sciences Unit, University of Glasgow}
%%\address{}


%% or include affiliations in footnotes:
%%\author[mymainaddress]{MRC/CSO Social and Public Health Sciences Unit, University of Glasgow}
%%\ead[url]{www.sphsu.mrc.ac.uk}

%%\author[mymainaddress]{Eric Silverman\corref{mycorrespondingauthor}}


\address[mymainaddress]{MRC/CSO Social and Public Health Sciences Unit, University of Glasgow, 200 Renfield Street, Glasgow G2 3QB, UK}
\address[mysecondaryaddress]{Institute of Health and Wellbeing, University of Glasgow, Glasgow G12 8RZ, UK}
\address[mytertiaryaddress]{Department of Computer Science and Information Systems, Teesside University, Middlesbrough, TS1 3BX, UK}

\begin{abstract}
Research in population health has been highly successful over the decades, as evidenced by its contribution to ever-increasing lifespans, low infant mortality, and other significant health improvements throughout the developed world.  The most troublesome population health challenges currently are often driven by social and environmental determinants, which are difficult to model using traditional epidemiological methods.  Here we make the case for attacking these challenges using complex systems simulation methods, particularly agent-based modelling (ABM). While ABM has been used occasionally in population health, we suggest that ABM has been unsuitably compared on a like-for-like basis with conventional epidemiological approaches, when the methodology is better-suited to answering different kinds of research questions.  We propose that these methods are a natural fit for the study of complex population health issues, and identify some key areas of enquiry where ABM can make a significant contribution to theory and to intervention development.
% * <u.gostoli@gmail.com> 2018-02-02T10:57:45.737Z:
% 
% > Research in population health has been highly successful over the decades, as evidenced by ever-increasing lifespans, low infant mortality, and other significant health improvements throughout the developed world. 
% Isn't this statement a little bit too bold? Although research in population health has its role, the overall improvement of the health indicators (lifespans and so on) is probably the result of a complex network of causes (economic, biomedical and so on).
% * <thorsilver@gmail.com> 2018-02-02T16:52:35.963Z:
% 
% > Isn't this statement a little bit too bold? Although research in population health has its role, the overall improvement of the health indicators (lifespans and so on) is probably the result of a complex network of causes (economic, biomedical and so on).
% 
% Yes I think you're right -- for now I've changed that to read 'as evidenced by its contribution to...'
% 
% ^.
% 
% ^ <u.gostoli@gmail.com> 2018-02-02T11:26:57.722Z.
\end{abstract}

\begin{keyword}
Agent-based modelling \sep Population health\sep Complexity \sep Simulation
%%\MSC[2010] 00-01\sep  99-00
\end{keyword}

\end{frontmatter}

\linenumbers

\section{Introduction}

%MMcC new draft 14/2/18 - Start

There are few settings where a deduction from a model will directly determine a policy decision; these settings tend to be those with low uncertainty, low risk, and where competing views regarding the situation at hand are absent. None of these apply when our focus is the distribution of health in populations, and the factors that contribute to it. A linear model - from scientific evidence to applied decisions - oversimplifies the multiple and competing interests, values, and outcomes for stakeholders which have an input into any action that will ultimately be taken \citep{oliver2014}.

%The UK government actuaries department (REF) provided guidance on the use of models (of any type) for government decision making. It highlighted two considerations we consider key. Firstly, there is often a disconnect between the focal points of modelling experts (specification, design, and model building, the areas on which methodological training focuses) and the focal points of greatest applied value (model interpretation, communication and decision making). Secondly, the extent to which models incorporate uncertainty, flexibility and risk will have a major impact on how those models may influence decision making. 

Causal inference methods (CIM) have proliferated in 21st Century population health research. Critics of the CIM approach have called for alternative conceptualisations of cause that are not solely based on probabilistic statements about population outcomes in alternative worlds \citep{krieger2016}. Decisions over causality can be based on pragmatic pluralism \citep{vandenbroucke2016}, or inference to the best explanation \citep{lipton2003,krieger2016} (often characterised as a form of abductive reasoning).  In all cases, causal conclusions are reached that fall beyond the deduction from a single model, source of evidence or theoretical frame.

The abductive approach falls closer to the conceptual process underpinning policy decision making, with some notable differences. While a scientist may continue to study - accumulating further evidence until they can abduct a conclusion - decision makers must often choose to act, accepting that evidence is incomplete and that the exact causal process remains uncertain.

In this paper we propose that greater use of Agent Based Modelling (ABM) approaches will benefit decision-making to improve population health. ABM projects can better complement the non-linear process through which science relates to action, change the model communication / model development split among those involved in modelling projects, and open up new opportunities for incorporating pluralistic views, and uncertainty in models. Taken in combination, the appropriate use of ABMs to complement existing research methods will improve the ability of public health scientists and policy makers to undertake abductive reasoning to inform decisions.

%MMcC new draft 14/2/18 - End

\section{The need for a new approach to decision-making}

%An effective intervention successfully disrupts a pathway between a cause and an effect. While there has been a long recognition of upstream social determinants of health, most historical successes in public health have focussed on modifying proximal causes, where a clear causal link has been postulated and upon which an intervention aims to exert influence. 

Population health has to its credit a number of major success stories over the years, not least a steady increase in life expectancy and major progress on serious health problems such as lung cancer and heart disease, due to concerted action on communicable disease through, for example, improved sanitation and mass immunisation programmes, and more recently on noncommunicable diseases through tobacco control and lifestyle education.  However, `despite major investment in both research and policy, effective action to tackle pressing contemporary public health challenges remains elusive' \citep{rutter2017}. The health problems which remain the most intractable are those that spring fundamentally from more complex causes: behavioural and social influence; and environmental interaction.  Some notable `wicked' problems include obesity, alcohol and drug misuse, and health inequalities, among others.

The `wicked' health problems of the 21st Century are characterised by numerous influences where the causal links are not clearly defined, and the mechanisms that influence them are elusive \citep{rittel1973}. Wicked problems in health are serious challenges for policy-makers, given that such problems ``are continually evolving; have many causal levels; have no single solution that applies in all circumstances and where solutions can only be classified as better or worse, rather than right or wrong'' \citep{signal2013}.  Behavioural risk factors underpin these wicked health problems, but how physical and social environments influence health behaviours, and what can be done to improve them is an area where the evidence is sparse. 

Identifying, implementing and evaluating effective responses to major population health challenges requires a wider set of approaches beyond the traditional methods of public health research and should involve a wider set of actors beyond the health services \citep{academy2016}. Population health is influenced by multiple, interacting determinants, including social, political, environmental, biological and behavioural factors; current challenges to the health of the public, and the salient overarching of inequalities in health within and across populations, are resistant to simple, linear, silver-bullet approaches \citep{rutter2017}.

These types of population health challenges, in which determinants of illness and disease are multifaceted, inter-related, and non-linear, are precisely where traditional epidemiological methods have the most difficulty \citep{galea2010}.  Unraveling the complex interactions of social and environmental determinants is difficult when using statistical methods in which individuals and their actions may not be explicitly represented.  Individual-based modelling practices, such as microsimulation, can model some of these behavioural and even spatial effects, but the effects of interactions between individuals and environmental interations are still difficult to capture.

We propose that complex systems simulation methods, in particular various forms of agent-based modelling (ABM), are a critical component in the fight against the `wicked' health problems facing us in the 21st century.  ABM provides a technique to better understand diverse influences on population health and thus should become a key part of the toolkit for tackling public health problems.  Ultimately, we take the view that decision-making related to complex, `wicked' health problems must make use of simulation in order to address these issues in a coherent way. 

In the remainder of this article we will summarise the current state of play in ABMs for health, critique their application and the methodological debate surrounding their use, describe issues that have been missed or overlooked among the ABMs for health community, and set out recommendations for the use of ABMs and the practice of research to improve the health of the public.

%We will detail the areas in which ABMs and related conceptual frameworks can supplement the work already being done in the field, and will provide examples of current and ongoing work which is developing ABM methods specifically tuned for population health applications.  In doing so we hope to provide inspiration and motivation for further work in this area, and develop the foundations of a complex systems science methodology for population health research.

\section{Computational Modelling in Population Health}

Despite the acknowledged promise of ABM and complexity-inspired approaches amongst some population health researchers, there are relatively few papers putting these methods into practice within the field.  Systematic reviews dig up mere tens of papers, despite ABMs having been used to study human social systems more generally for nearly 50 years \citep{nianogo2015}. 

Clearly the uptake of ABMs in population health is currently minuscule, and well below the prominence of other methods such as multilevel modelling.  We suggest there are three main factors driving this lack of enthusiasm for the approach. Firstly, claims of what they can offer above traditional methods are contested. Newer statistical models may be capable of estimating causal effects in the presence of dynamic processes, treatment interference or spatial and network autocorrelation \citep{naimi2016}, thus removing the justification for choosing ABMs over an alternative approach with which the community is already skilled. 

Secondly, computer programming, simulation modelling and complexity science are not routinely taught to the feeder disciplines for population health practice and research, unlike  statistical theory, data management, and quantitative analysis.  This is due to the traditional epidemiological focus on Positivist hypothesis-testing approaches to identify causal processes and intervention effectiveness.  Adoption of ABMs, in contrast to novel statistical techniques which require an incremental adjustment to people's thinking within the same paradigm and build on existing skill sets, may require substantial retraining and changes in the way in which people think about and characterise population health challenges.

We suggest there is a third factor, one worth examining in detail: ABMs within population health are frequently compared like-for-like with approaches designed to answer fundamentally different research questions.  As a consequence, the methodology is perceived as lacking precision and tractability when compared to statistical methods, when in reality the ABM is simply better-suited for a different type of question.

\section{Challenges from complexity in public health}

We claim that the study of `wicked' health problems necessitates the study of human society as a complex system. Complex systems in this context can be defined as systems composed of interacting adaptive agents. More precisely, the dynamics of a complex system are driven by the interaction taking place at the level of the micro-units of which it is composed, and the micro-units' adaptations to the environmental changes they mutually generate. As such, social systems display some characteristic properties which do not lend themselves to reductionist approaches, but instead necessitate the adoption of `bottom-up' models, i.e. models based on an explicit representation of the behaviour of the micro-units composing the systems -- in this case, individuals; the direct interactions between the micro-units; and interactions between the micro-units and their environment. 
  
These properties, moreover, suggest that public health can be conceptualised as the aggregate outcome of a complex system.  That being the case, only by modelling individual behaviours and their interactions (through ABMs and similar methods) can we develop models allowing us to study the tangled web of causal relationships among environmental, physical and social factors affecting health-related behaviours and, at the aggregate level, their effect on the dynamics of public health problems.

Below we explain the characteristics of complex systems, the ways in which human society fulfills these characteristics, and the resultant impact on research efforts on such systems. 
 
\subsection{Emergence}
ABMs, and complex systems more generally, historically have been deeply connected to the concept of emergence, which is often divided into strong and weak forms.  Frequently, references to emergence are referring to \emph{strong emergence}, which describes properties of systems which are not deducible from the behaviour of their component parts.  Unfortunately strongly-emergent phenomenon are not particularly amenable to simulation, as the simulation would be a reduction of the system to its component parts and therefore would not gain us any further insight into a property which by definition is not reducible to the actions of those parts.

%In the context of the development of complexity science and the subsequent popularity of ABMs, modellers and philosophers alike have devoted considerable time to the exploration of emergence and its consequences.  
Consequently, Bedau's exploration of \emph{weak emergence} has become an important concept for agent-based modellers: a property is defined as weakly emergent if that property is unexpected given the rules that govern the system's component parts (i.e. the system's micro-units), even if those rules are well-understood \citep{bedau97b}.  Bedau suggests that these properties are particularly amenable to being simulated; Chalmers illustrates this using the example of cellular automata, in which simple, easily understood calculation rules can produce unexpected higher-level patterns \citep{chalmers2006}. In this context, the term 'complex' has a relative meaning and also brings to light our cognitive limitations -- our inability to run mental simulations allowing us to forecast the system's dynamics, even when it runs based on relatively simple rules. In other words, due to our limited cognitive capabilities, we are unable to qualitatively forecast the aggregate outcome of the interplay between the system's micro-units \citep{berkman2011}.

A well-known example of emergence is given by Schelling's residential segregation model \citep{schelling71, schelling78}. In this model, very simple agents express a preference for the group composition of their neighbourhood by choosing to move when the number of their neighbours from a different group exceeds a certain threshold.  Here the agents' segregation is the emergent property of the system, while the preference for segregation is the parameter driving the agents' behaviour. Schelling showed that even a relatively low preference for segregation (e.g. the agents' preference for not being in the minority in their neighborhood) generates a high degree of residential segregation, a result which is not predictable solely by knowing the agents' behavioral rules.  Thus we can describe Schelling's model as \emph{weakly emergent}, given that it displays unexpected properties despite its known, and very simple, behavioural rules.

When modelling public health, emergent properties should be a key part of the conversation.  Successful interventions may seek to alter individual, low-level behaviours through a number of different routes, in the hope that the population-level picture which emerges from those actions changes for the better.  Following Axelrod and Tesfatsion, we might align ourselves to using ABMs for normative understanding, or `evaluating whether designs proposed for social policies, institutions, or processes will result in socially desirable system performance over time' \citep{axelrod05}.  Agent-based modelling, as a methodology tailored to the investigation of emergent phenomena, is well-placed to provide insight into these situations by not only explicitly modelling individual behaviours in response to an intervention, but also interactions between individuals and with their environment.

\subsection{Non-linearity}
In describing complex systems and their behaviour, we need to distinguish between two different acceptations of 'non-linearity' of complex systems. The first kind of `non-linearity' refers to the kind of causality which characterizes a certain process. In a model characterized by causal linearity, causation is directed in one single direction. System dynamics has repeatedly shown that, for many real-world systems, linear causal models are the exception rather than the rule, as these systems are often characterized by cyclic, relational and mutual causal relationships among the variables describing the system's state. An additional kind of non-linear causal relationship, which is specific of systems composed by interacting micro-units, may be called \textit{diffuse causality} and it refers to the process through which the behaviour of a micro-unit indirectly affects the behaviour of $N$ other agents via its effect on the agents' local environment -- a `chain reaction' through which the initial effect propagates through the system. The Schelling model again offers a clear example of this kind of non-linear causal relationship: an agent moving to another location changes the local environment of both the old and the new location's neighbors, affecting the probability that they will too move to another location. 

Complex systems are characterized by a tangled web of non-linear causal relationships, which makes the distinction between exogenous and endogenous variables a blurred one. The non-linear causal relationship among the micro-units and between the micro-units' and their environment is the fundamental reason behind our incapacity of forecasting the system's dynamics from the micro-units' behavioural rules: the web of causal relationships is simply too tangled for our limited cognitive capabilities to take all of them into account when trying to run a mental simulation of the system. 

The second definition of 'non-linearity' refers to the kind of relationship between one or more exogenous variables and one endogenous variable: in this case, saying that the relationship is not linear means that variations in the exogenous and the endogenous variables are not proportional. In the Schelling model, we may change the agents' tolerance for different neighbors without noticing any significant change in residential segregation at the population level, as long as we are below or above a threshold level. Once we reach this level, however, a slight change in the agents' tolerance changes the system from a mixed state to a segregated one. Thus, the relationship between the agents' tolerance and the system's level of residential segregation is a non-linear one. Of course, these two kinds of 'non-linearity' are strictly related. It is because complex systems are characterized by non-linear causal relationships between the micro-units' behaviour that often we observe a non-linear relationship between exogeneous and endogeneous variables at the aggregate level.\\

\subsection{Adaptive behaviour}
Adaptive behaviour refers to the capacity or propensity of an agent to change its state following a change in its environment (which includes the agent's neighbors). As such, this is a fundamental characteristic of a complex system, as it allows for non-linear causality in these systems: an environmental variation prompts the agents' behavioural responses, which then feed back into additional environmental variations, and so on. In other words, in a complex system, the feedbacks runs through the micro-units, which may affect each other both directly and indirectly, through changes in their common environment. 

Being the micro-units the `engines' of the co-evolutionary process driving the system's dynamics, the model defining the behaviour of these micro-units is the fundamental building block of any complex system's model: we cannot account for the dynamics of the system unless we explain it as the result of the interaction between the system's micro-units and their environment.

Human societies are undeniably characterized by adaptive behaviour of the most complex kind, as human beings are able to recognize the system they are part of as a complex one, identify the system's emergent properties and develop models that take them into account to drive their own actions. As Gilbert put it:

\begin{quote}
In the physical world, macro-level phenomena, built up from the behaviour of micro-level components, generally themselves affect the components....
The same is true in the social world, where an institution such as government self-evidently affects the lives of individuals. The complication in the social world is that individuals can recognise, reason about and react to the institutions that their actions have created. Understanding this feature of human society, variously known as second-order emergence, reflexivity and the double hermeneutic, is an area where computational modelling shows promise \citep[][p. 5]{gilbert00}.
\end{quote}

This phenomenon of second-order emergence, or the fact that emergent social institutions become part of the agents' models driving their behaviour, creates direct causal relationships between the micro-units' behaviour and the system's dynamics, which further compounds the complexity of the system.

\subsection{The challenge to traditional epidemiological approach from complex systems}
Having outlined the defining characteristics of complex systems, we can now better understand why they pose a challenge to the statistical approach typically adopted by epidemiology and how ABM can help epidemiology to rise to this challenge. Public health problems can be seen as the emergent outcomes of the complex social system that is human society. As such, to understand their dynamics we need to develop a model based on the explicit representation of the micro-units that compose this complex system -- individual human beings.  The agents' adaptive behaviour and the resulting complex web of causal relationships between agents and their environment make the non-linearity of the relationship between the systems' variables a pervasive feature of human society. 

The implication is that a very small variation of system inputs can generate a big variation in system outcomes, or vice versa. 
We can visualize a non-linear complex system as one where the space of possible outputs is very rough along the many input dimensions: because of the number of factors affecting the relationship between any two variables, points that are very near to each other in the space of any input, can be far apart in the space of outcomes. Now, while the traditional statistical approach can be used, \textit{in principle}, to shed light on the causal mechanism through which variable X affects variable Y (and in fact much of the epidemiological research consist of the addition of confounders and mediators to the original theoretical model to enhance our understanding of how variable X affects variable Y), it relies on the availability of a `sufficient' number of observations for the analysis to have enough statistical power, a number which increases with the number of confounding variables which are part of the causal model.
This represents, \textit{de facto}, a limit to the complexity of the theoretical model which can be statistically analyzed. In this regard, there are two main problems. First, a complex system may contain variables for which it is difficult (or impossible) to gather empirical values and which, therefore, cannot be controlled for. But even if our theoretical model does not contain such variables, a second problem is that in complex systems the number of \textit{potentially} conditioning variables is typically very large, so that we may easily end up with having too few observations to conduct a meaningful statistical analysis of the relationship between the variables of interest.
This is the reason why most of the causal models underlying the traditional epidemiological approach are quite simple (simple enough to be based, in many cases, on an implicit causal mechanism): they need to be that way in order for the number of observations to be large enough for the analysis to have the desired level of statistical power. In other words, the traditional epidemiological approach is forced, by the tools it adopts, to assume that many variables, which we may want to include in our model \textit{in theory}, do not affect the relationship between X and Y, an assumption which we may call the \textit{stability} assumption, as it requires that the relationship of interest is not affected by changes in other, `contextual', variables.

When data is scarce, as it is often the case with human complex systems, the statistical approaches become rather helpless, but properly specified theoretical causal model can still be usefully used to increase our understanding of the system's behaviour, through sensitivity analysis and the simulation of `what-if' scenarios. Indeed, we claim that ABM may be a very effective methodology to develop such theoretical models when investigating complex systems. With ABM, we formally codify our theoretical knowledge of the system in a computational model which produces (through simulations) counterfactuals, allowing us to assess which contextual variables we may \textit{not} consider in the list of conditioning variables, the extent to which the stability assumption is tenable and, if not, which kind of effects on the relationship of interest we may expect from the conditioning variables. Therefore, rather than as a substitute of the traditional approach \textit{tout court}, ABM can be more properly seen as a complementary tool to assess the limits of the statistical approach when it deals with variables which are part of complex systems and, more generally, to investigate the behaviour of these systems in situations when the scarcity of data warns us against the adoption of statistical approaches.

% * <u.gostoli@gmail.com> 2018-02-02T14:02:48.458Z:
% 
% Both the first and the second to last sentence refer to the concept of 'emergence', which should be the topic of the last point (3.3), and this makes this paragraph a little confusing. Perhaps, the relationship between non-linearity and emergence needs to be highlighted.
% With regards to the debate about the suitability of the ABM approach to model complex systems (and the advantages of this approach compared to the traditional linear models), in my view,  
% In fact, point 3.2 below this, focuses just the fundamental reason that models of the human society cannot be adequately described by linear causal models: adaptive behaviour. Indeed, 
% When cyclical or mutual causation is the rule, the estimation through traditional statistical models requires the assumption of 'equilibrium' and 'stationarity' of the system, assumptions that cannot be directly verified in the data. 
% 
% ^ <u.gostoli@gmail.com> 2018-02-02T14:59:13.511Z.

%A key feature of complex systems is that observed system-level phenomena cannot be explained by focussing on the individual level units from which these phenomena emerge. Once again Schelling's residential segregation model provides an example: the model demonstrates that the appearance of stark residential segregation at the population level is caused by individual's housing preferences, even when individual preferences for segregated living are very weak \citep{schelling71,schelling78}. This model was not used to determine the causal effect of neighbour similarity on likelihood of house relocation; conceptualising segregation derived from individual behaviours was the outcome. It is not difficult to think of applications in public health that similarly do not invoke effect estimates. Inequalities in health behaviours, geographical patterning of health, and social patterning of healthy / unhealthy communities are of inherent interest to health policy.

%Naimi's critique of the ABM approach is that the standard epidemiologic approach can identify causal effects in the presence of nonlinearity, and these considerations do not necessitate an ABM \citep{naimi2016}. If we are only considering how a deductive versus generative model provides a causal effect estimate, this critique holds some weight. However, ABMs facilitate the understanding of systems, not simply the estimation of effects. Emergent properties are a key feature of our complex world. The persistence of inequalities in health, the entrenchment of behaviour patterns over generations, and in more recent years, viral media influences on health such as Neknomination are examples of emergent properties at the system level; epidemiological effect estimates do not characterise these patterns and are insufficient to study them.

%\subsection{Adaptive behaviour}

%Another important feature of the relationship between individual behaviour and system level patterns in social systems is the extent to which individuals adapt their behavior to a changing environment. The pace of biological evolutionary change makes the assumption of a stable causal effect of an exposure on a disease outcome tenable in relation to physical disease processes. This assumption does not hold for social behaviours relating to health. The diffusion of innovations model demonstrates how individual decisions may be strongly influenced by the number of others who have made that decision previously (REF); and survey fatigue - the reducing likelihood of positive response, dependent on previous responses - has long been recognised as a concern for social research. 

%ABMs are capable of modelling adaptive agents, simulated individuals whose responses to their environment change over time. Such an approach provides an understanding of non-linearity which goes beyond those offered in Naimi's critique; rather than a non-linear distribution of effect for given levels of the exposure, adaptive behaviour implies a changing causal effect within individuals over time. Methods such as genetic algorithms provide an \emph{in silico} approach to modelling adaptive behaviour.  We propose that `wicked' problems in public health require better methodologies for understanding how adaptive behaviours - or non-linear and changeable causal effects - lead to patterns of health risk behaviour and population health outcomes.


% * <u.gostoli@gmail.com> 2018-02-02T15:14:24.310Z:
% 
% Non-linear causal processes (in the second acceptation of the term, described above) and, in particular, adaptive behaviour, generate emergent phenomena. Perhaps, the usage of the expression 'emergent behaviour' may be confusing as here it refers to a property of the system, while in the previous point we talked about individual behaviour  (i.e. adaptation). 
% A prototypical example of emergence is, of course, the emergence of segregation in the Schelling model, mentioned in the first paragraph of this section. 
% In my view, this paragraph could be the occasion to talk about the importance in the modelling of complex systems (besides multi-directional causality, as said above), of the modelling of the behaviour of the micro-units the system is made (the people, in the case of models of human society). While traditional epidemiological approaches propose statistical models of the relationship between aggregate variables, in complex systems the dynamics of aggregate variables depend on the interactions taking place at the level of the micro-units and the feed back process between the environmental change and the micro-units' response. Of course, even in complex systems we can find some kind of statistical relationship between macro variables. The problem is that without an understanding of the micro-foundations of this statistical relationship, we cannot use the statistical model to effectively inform a health policy intervention. The Schelling model can be useful to illustrate this point. Let's suppose that we have a measure of the segregation in different societies (the 'endogenous' variable) and a measure of the preference for segregation in these societies (the 'exogenous' variable). Now, a statistical analysis would reveal that the two variables are not related: segregation characterizes both the societies with a high and those with a low preference for segregation. This, for a traditional epidemiologist, would be the end of the story: his conclusion would be that if we want to tackle the problem of segregation we have to look somewhere else than the preference for segregation. In fact, the Schelling model tells us another story: the preference for segregation does depend on the preference for segregation and it is, in fact, the only behavioural parameter of the model. However, the mutual causality between the environment and the agents' behaviour generates a high level of segregation even for low levels of individual segregation's preferences. So, in order to understand segregation we need to understand its nature of emergent phenomenon, that is, we need to understand (and have a model of) the micro-process from which it emerges (and the bi-directional causal relationship characterizing this micro process). 
% 
% ^ <u.gostoli@gmail.com> 2018-02-02T16:08:13.174Z.



\section{A new paradigm for complex public health}

While ABMs offer great promise for the public health community, we suggest that their implementation is currently focussed too heavily within the mainstream paradigm of traditional epidemiological methods. As explained above, ABMs can have important and distinct uses in public health, separate from the uses of traditional methods. Below, we outline a number of ways in which ABMs could support the public health community.

ABMs offer a method to understand how the combination of agent interactions and non-linear causal effects give rise to system level patterns. Such an approach falls outside the traditional epidemiological toolkit, but developing our understanding of these patterns is of essential importance for improving the health of the public. With this in mind we would concur with Auchincloss and Diez-Roux, who outlined how regression approaches can limit our enquiries:

\begin{quote}
Too often, the exclusive use of regression approaches constrains not only the answers we get but also the types of questions we pose and the hypotheses and even theoretical explanations that we develop. In our search for what is `tractable' in empirical observational research (essentially that which mimics the perfect experiment), our questions have the tendency to become narrower and narrower and perhaps less relevant to understanding or intervening in the real world \citep[][p. 6]{auchincloss2008}.
\end{quote}

Eight years later, the methodological debate continues to discuss the relative merits of ABMs compared to regression in the presence of interference or dynamic processes.  ABM researchers are no strangers to these debates -- the case of demography provides an illustrative example of how adopting ABMs requires not just a change of methods, but a change of mindset and conceptual framework.

\section{Model-Based Science: The Case of Demography}

The recent evolution of modelling methods in the population sciences illustrates how theory-driven, model-based methods can alter a discipline and its approach to data.  Courgeau et al. (2017) take stock of the progress of demography over the centuries, and propose that the discipline is on the cusp of a new methodological paradigm.  They note that the progress of demography to date has been a cumulative journey -- each new methodological revolution has enhanced the previous, and yet no new methodology has completely replaced any previous one.  Each methodological tool still has its place in the metaphorical toolbox. 

Agent-based modelling, however, does not fit this pattern as cleanly.  As it is not a statistical revolution, but a simulationist one, agent-based methods cannot simply be added to the methodological toolbox and applied to the same problems in the same way.  When multi-level analysis was developed, previous data could still be used and the methods guided demographers about what new data would be most useful.  ABMs, however, have a more complex relationship with data, and in general what keeps the modeller awake at night is not a lack of data but instead the need for sensible parameter values.
So the ABM researcher in demography needs to think not only about data sources, but some new topics, such as:
\begin{itemize}
    \item Social and behavioural theories: what should our agents be doing, and why?
    \item Uses beyond prediction: what can the ABM add to our knowledge of population change?
    \item Different usage of data: suddenly we may start using qualitative data, finely-detailed GIS information, etc.
\end{itemize}

For a statistically-focused discipline like demography, the move toward model-based science is a complex one.  It requires not only knowledge of a new approach and its epistemological and computational limitations, but a new perspective on how and when the approach should be used, and where it should fit in the overall scientific spectrum of population studies.

We propose that the relationship between ABMs and population health needs to change in a more substantive fashion before their usefulness becomes apparent.  Much like in demography, the development of a theory-based, model-centric approach to certain problems is needed, and in so doing we will uncover areas where ABMs are more suitable.

In demography, these problems have been focused on explanatory aims, and most commonly applied to areas where simply `throwing data at the problem' is not proving successful.  Migration, for example, is commonly seen as the most uncertain aspect of population models in demography, so ABMs have been applied to migration to examine possible motives for migration behaviours in various contexts (Willekens 2012, Klabunde 2014, etc.).  In the context of public health, ABMs can be applied particularly well to areas where sociality is influential in determining the success of interventions. 

\section{Roles for Simulation in Population Health}

As described above, some of the most challenging, `wicked' problems in population health involve complex, interacting processes that are difficult to characterise in a traditional epidemiological framework.  These challenges offer an opportunity for ABMs to contribute to our efforts, in that they can explicitly model the social and environmental aspects of population health issues.

However, this still begs the question: what will such models actually do?  How would our quest to reduce or eliminate a `wicked' problem benefit from an ABM, and how would it add to our knowledge?  Here we illustrate a few ways in which ABM approaches can enhance our efforts.

\subsection{Models as Policy Sandboxes}

Evidence for population health policy is frequently based on causal inference derived from health technology assessment methods. Health policy making frequently considers factors other than evidence on causes. Given the evidence of poor transferability of interventions, looking to other factors seems a sensible approach, as much as it causes difficulties for demonstrating research impact.  

ABMs offer the opportunity to build a policy sandbox; a place to explore what causal evidence would be useful for taking a certain decision, to develop collaboratively a theory for why a certain policy will or will not work, and to test out how interventions could have an effect, if these theories hold true. Di Paolo, Noble and Bullock (2000) propose that while ABMs are often approached as a method to develop `realistic simulacra' of the physical world, they may be best placed as platforms for the exploration of theoretical relationships within a system, their interactions and consequences. Such conceptual exploration can be very useful for building theory. An approach that facilitates the apprehension of concepts in an abstract, rather than data-driven sense is useful for the precise reason that it differs from the data-driven approach of orthodox epidemiology. 

In essence, traditional epidemiology focuses on risk factors and outcomes, with the link between them being an opaque black box \citep{susser1996}. However, in order to develop interventions to change the relationship between risks and outcomes, what is within that black box itself needs to be changed. This requires understanding the mechanisms and having a theory of change in order to modify them. Statistical methodology does not provide the tools to model the mechanisms, so researchers tend to focus on tasks that statistical methods can solve, which is the description of inequalities identified by the data without necessarily providing solutions. ABMs in contrast require people to model mechanisms explicitly, at least in the abstract, and provide a set of tools which encourages people to focus on the parts of the system where change might occur and could have the greatest impact. Thus creating more solution-focused research.
% * <u.gostoli@gmail.com> 2018-02-02T16:22:27.680Z:
% 
% This is a fundamental point, and I think that as been well illustrated here.
% Estimating the statistical relationship between two or more variables shouldn't be confused with a theory of the relationship between these variables. The statistical model measures (at best) the effect of one or more exogenous variables on one endogenous variable, whereas a theoretical model explains the effect, that is, proposes a story of why and how a set of variables affect the outcome of interest. In fact, underlying any statistical model there is an implicit causal theory which is assumed to represent the reality by the statistical analyst. The problem arise when we deal with complex systems because hardly we can store  a faithful representation of the real system in our brain (let alone running simulations of the model), given our cognitive limitations. We have to make the theory explicit by developing an agent-based computational model, and to draw the model's conclusions by running simulations. 
% In this process, the traditional epidemiological approach does have a role: the model's simulations must be consistent with the statistical regularities found by the statistical analysis. In other words, the traditional approach, can help to validate and calibrate the agent-based models.
% However, the ABM, once it is consistent with the empirical data and the statistical relationships found through empirical analysis, does add to our knowledge, as it defines the contextual conditions under which particular statistical relationships hold.  In order words, by understanding the process underlying a particular statistical relationship between two or more variables, we can make deductions (through simulation) about how this statistical relationship would change under alternative scenarios or policy interventions.
% 
% ^ <u.gostoli@gmail.com> 2018-02-02T16:52:54.074Z.

As Marshall notes, ABMs also provide a means to make use of a wider range of evidence: 

\begin{quote}
Agent-based modeling represents one (but not the only) method to synthesize prior knowledge of a population -- and the causal structures that act on this population -- to understand how an intervention could affect the public's health. In this manner, agent-based modeling is a science of evidence synthesis. Specifically, ABMs (and other simulation approaches) represent a platform for the integration of diverse evidence sources, including inconsistent or inconclusive scientific information, to support decision making for complex public health problems. \citep{marshall2017}
\end{quote}

In this respect we can be inspired once again by the field of demography.  The ability of ABMs to serve as theoretical exploration tools and intuition pumps even in the absence of data has been seen as an advantage in some circles \citep{silverman2011}, as demographers would be able to investigate theories about population change using ABMs even in the absence of expensive, difficult-to-collect survey data, etc., upon which statistical approaches depend.  The rise of ABM has even been presented as the appearance of an altogether new approach to demography as a model-based science -- one in which the study of rich interactions within a population form a central part of the understanding of population trends at the higher level \citep{corgeau07, courgeau2017}.

These same benefits could be significant for population health research.  Health data can be expensive to collect, as well as legally challenging due to strict data protection laws.  Being able to circumvent these problems and use generative approaches to understand complex health problems would allow for potentially fruitful investigations of potential policy interventions even in the absence of detailed data.  Further, the outcomes of our simulations may help us identify where future data collection is needed.

\subsection{Investigating Assumptions}

ABMs have provided a method to run in silico interventions to inform policy makers about options, based on counterfactual runs of models on agents behaving under the same rules in different environments. They could go further, and allow policy makers direct input into which elements of intervention theory are more  or less plausible in their local context, to modify the generative mechanisms that underpin the behaviour of agent in the system and change the specification of a causal theory. 

Traditional analysis of data is agnostic to the audience. ABMs in contrast provide a means to conduct post-normal science, or a scientific enterprise conducted in a context where `the puzzle-solving exercises of normal science (in the Kuhnian sense) which were so successfully extended from the laboratory to the conquest of Nature, are no longer appropriate for the resolution of policy issues of risk and the environment' \citep[][p. 750]{funtowicz1993}.  In post-normal science, decisions with huge stakes must be taken in an environment where the scientific knowledge available is unavoidably uncertain, and as a consequence input from the public and from policy-makers becomes critical. Generative theories of causation appropriately privilege the body of scientific knowledge, but also integrate with, and adapt in response to the expertise of knowledge users. This provides a paradigm where evidence can inform policies to improve health, moving beyond a context-free evidence approach. 

This does not mean that the science yields to opinions, values or political will of knowledge users - all the principles of verification, validation and calibration are still integral, but now the ABM provides direct comparison on the validity of in silico systems between those that do and do not incorporate broader knowledge. A fruitful avenue for future research is studying the extent to which human-calibration versus data-calibration produces models which A.) can usefully inform health policy work and B.) produce effective policies.

Rather than setting our minds to the task of how ABMs can better draw on the evidence base,  ABMs provide a mechanism of reaching a collaborative agreement (drawing upon various forms of evidence and input) over how interactions in social and spatial environments give rise to population patterns. An ABM provides body- and space-syntonic feedback on the validity of shared assumptions underpinning a population health issue. Differing assumptions between research evidence and evidence users can be investigated using parameter space exploration.

\subsection{Parameter space exploration}

One area in which ABMs remain opaque in some respects is in the analysis of results.  ABMs by their nature are suitable for modelling the behaviour and evolution of systems which defy formal statistical analysis, and that very complexity means they can appear opaque, where inputs and outputs are present but the influence of processes within the model are tangled and unclear.

However, recent advances in uncertainty analysis and the advent of machine learning for the processing of `big data' opens up new possibilities which can make ABMs more transparent. For instance, machine learning has been proposed in combination with ABM to achieve automatic calibration of simple trajectory samples, where agent-generated movement is used as an input of a machine-learning pipeline \citep{torrens2011building}. 

Machine learning methods can improve the theoretical understanding of the ABM, help calibrate the model, and facilitate interpretation of the results relevant to end users, therefore achieving an overall model of complex health issues with greater empirical validity.
For parameter spaces that cannot be searched effectively with heuristics, a semi-supervised machine learning approach has been proposed based on surrogate models.  User-defined labels can be combined with the surrogate machine learning-based version of the ABM, to search the parameter space and calibrate the ABM, with a striking reduction of computational time \citep{pereda2017brief,lamperti2017agent}. 

After the construction of the ABM, a series of simulations and input-output data will be generated. A deep learning approach can be then used to create the surrogate model, through the construction of a multi-layered neural network. This network can be trained on the data deriving from the ABM simulations \citep{van2017deep}. As a result, with good accuracy and a dramatically reduced computational effort, the neural network is able to approximate the ABM model results, and can be used for further parameter optimization. Although the universality theorem states that for any continuous function there exist a neural network with a single layer that approximates the function to any desired precision, finding the optimal structure of the network in practice will require further investigation on the number of layers and neurons, as well as optimising the hyperparameters of the network and its learning algorithm.

Configuration parameters, as well as ABM simulation results, can also be used to train a machine learning method, with one or multiple targets to achieve \citep{furtado2017machine}. The method will therefore act as a surrogate of the ABM, and can simulate the ABM behavior for a much larger of combinations of parameters and in a shorter time, indicating parameter values that are likely to be of interest when explored directly in the full ABM. 

In the area of uncertainty analysis, the advent of the Bayesian Analysis of Computer Codes (BACCO) methodology has produced significant advancement in the analysis of model uncertainty and the impact of parameter values \citep{ohagan06}
In particular, Guassian Process Emulators (GPEs) have proven to be a promising method for analysing ABMs.  In essence, a GPE takes a training set consisting of simulation outputs resulting from a wide range of input parameter values taken from across the parameter space, then develops a statistical model of the simulation model \citep{kennedy01}.  GPEs assume that the output variable of interest can be decomposed into main effects derived from input parameter values, alongside a constant term, and the uncertainty introduced by the computer code itself (e.g., rounding errors).  The result is a summary of the fraction of the final output variance accounted for by each input parameter, as well as their interactions.  This gives us a much clearer picture of the impact of each input parameter.

GPEs have been applied successfully in ABMs of various types, including models of social care \citep{silverman2013soc}, research funding allocation \citep{silverman2016}, and the effect of landscape changes on bird populations \citep{parry2013}.  Much of this work has been facilitated by the availability of GEM-SA, a free software package for analysing simulation results using GPEs \citep{kennedy2004}.  As this technique continues to mature, and additional user-friendly means of implementation are created, variations of the GPE approach seem likely to become a common method for better understanding the internals of an ABM.

GPEs and related techniques can also be applied before the ABM parameter settings are finalised.  If particular parameter values are uncertain, or their values are unknown or difficult to estimate, modellers can use GPEs to develop a training set covering a wide portion of the parameter space, and then use the resultant sensitivity analysis to identify portions of the space that produce realistic values for the main output values of interest.  These areas can then be examined in greater detail with focused series of runs of the full simulation.  Thus, the GPEs can help identify reasonable values for unknown parameters, and reduce the computational expense of doing so by running the emulated model rather than the full simulation.

%\emph{--(triangulation)}

\section{Ways forward}

As we have seen, ABMs and concepts drawn from the study of complexity could have a valuable role to play in future research in population health.  The `wicked' problems of the 21st century present a significant challenge to traditional statistical approaches, so approaching them from a complex systems perspective can enable us to characterise these problems in terms of the social and environmental interactions which lie at the core of these societal problems.

As with any field, not every tool is fit for every task, and this remains the case with ABMs and related methods.  In the case of less `wicked' issues in population health where causal links are more clearly defined, statistical methods already available and easily implemented can work quite well to generate the kind of effect estimates health researchers might seek.  ABM should be a significant and powerful addition to the toolbox for population health modellers, used in harmony with the trusted epidemiological methods already in evidence.   

For efforts extending further into explanation and description of more complex problems, ABMs start to shine.  Below we describe three areas of potential interest population health researchers keen to experiment with ABMs, and note the difficulties facing traditional approaches in these contexts.

\subsection{Health Inequalities}

Health inequalities, which are systematic differences in the health status of different population groups, remain a difficult problem even in countries with otherwise high levels of equality \citep{bambra2011}. For example, Nordic countries, despite generous welfare states and generally excellent population health statistics, still have persistent health inequalities \citep{shaw2014}. Likewise, health inequalities have persisted in the United Kingdom even in the presence of free health care and systematic attempts by the New Labour government to reduce them \citep{mackenbach2011}.  

A potential reason for this persistence is the sheer level of social reform that would be required to tackle this problem \citep{mackenbach2011}.  Wealth inequalities might need to be corrected, access to healthcare and hospital/clinic locations might need to be changed and optimised, and significant educational efforts to encourage changed health behaviours might need to be undertaken, amongst other changes.  In the real world, policy interventions on this scale are not only cost-prohibitive, but quite disconcerting to policy-makers, who are (quite understandably) reticent to restructure the whole of society on the basis of theories which in effect cannot be implemented without tremendous risk of unintended consequences \citep{berkman2011}.  These unintended consequences, or spillover effects, can potentially impact negatively on other key areas of population health \citep{lorenc2014}. 

ABMs, however, provide a testbed of sorts.  Generative approaches and virtual populations can provide a risk-free laboratory for implementing large-scale policy interventions on simulated populations.  While we would not wish to claim these kinds of models necessarily provide strong predictive power, if given sufficient detail in relevant social and behavioural mechanisms they could provide important insights on potential unintended consequences of these large-scale policy interventions.

More broadly, ABMs can shed light on those areas where health effects at the population level appear to be at odds with our expectations, given our knowledge of individual-level behaviours and processes.  The case of inequality in Scandinavian welfare states above is a good example: we would expect very equal societies with universal access to free healthcare to show minimal health inequalities at the population level, but in reality health inequalities are both still present and surprisingly significant.  ABMs and complex systems approaches could help population health researchers unravel these mysteries by modelling the social and environmental effects that might lead to differential health outcomes despite society being set up specifically to avoid those outcomes.

\subsection{Alcohol Use and Misuse}

In the UK, the harm caused by alcohol both to the individual and to others is significantly higher than that caused by other drugs \citep{nutt2010}. The role of psychological \citep{hammersley2014}, social \citep{meier2018} and environmental factors \citep{birckmayer2004} in influencing levels of substance use and dependence are increasingly recognised as key factors beyond biological addiction processes, both in terms of the risk processes, and for the most appropriate points of intervention. Scotland has recently implemented a minimum price per unit of alcohol \citep{brennan2014}. 

Predicting the outcomes of population level interventions is far from straightforward. Price-based measures are more likely to be effective amongst drinkers on the lower end of the socioeconomic scale, whereas changes in pricing will have a greater impact on day-to-day spending habits \citep{holmes2014}, but availability of alcohol (at any price) is strongly patterned by geography, with deprived areas of Glasgow having a greater density of alcohol, tobacco, fast food and gambling outlets, for example \citep{macdonald2018}. How alcohol retailers respond to minimum price legislation in areas of high compared to low supply may influence, and be influenced by changes in purchasing behaviour. Traditional statistical approaches are limited in the extent to which they can study non-linear interactions such as those between individuals and local retailers.   

ABMs and complexity-inspired approaches can make substantive contributions in this area.  Statistical modelling studies have looked in depth at minimum unit pricing policies, but computational models could allow us to investigate how interactions between individuals, individuals and their neighbourhoods, and  related effects such as socioeconomic differences \citep{holmes2014}.  Alcohol and drug abuse treatment outcomes are also affected by social factors \citep{schroeder2001,homish2008}, and modelling these interactions and networks via ABMs could help in further fine-tuning interventions in this respect \citep{yorghos2018}

\subsection{Obesity}

Obesity is the target of substantial research and investment in population health worldwide, given that the condition is linked to such a wide range of health problems -- both as an effect and a cause.  Obesity stands out amongst the target research areas listed here, in that a number of simulation studies have already targeted obesity, and there have been some high-profile calls for systems-based modelling of the problem.  The Institute of Medicine presented a book-length study of obesity prevention efforts in the United States, concluding that researchers and policy-makers must take a `systems thinking' perspective \citep{IOM2010}.  Chapter 4 provides a useful look at systems perspectives on the obesity epidemic -- useful background for interested modellers \citep{IOM2010}.  Similarly, Foresight in the UK was commissioned to develop a strategic 40-year plan to tackle obesity in the UK, and presented a whole-system approach to the problem advocating large-scale societal, personal, and governmental changes \citep{Foresight2007}.

Skinner et al. outline nine properties which make obesity well-suited as a target for systems-science-based modelling efforts, and present a systematic review of systems science models in obesity research (\citep{skinner2013}).   While their review finds dozens of studies using systems science perspectives and techniques, they conclude that the work done thus far has taken relatively limited views of the topic, investigating manageable portions of the overall obesity problem rather than tackling the larger, messier complex of societal and individual factors which drive obesity trends.  Levy et al. take a similar view, noting that obesity modelling efforts are `at a nascent stage of development' and just focus on `one or two links in the process', leading the authors to advocate a comparative modelling approach (\citep[][p. 390]{levy2011}). 

Obesity is thus a highly relevant area for modellers of an complex systems background, and indeed work of this type is already underway \citep{silverman2017}.  Much of the conceptual work around the social, behavioural and environmental factors influencing the obesity epidemic has already been done by respected sources.  The widespread agreement that ambitious societal change at multiple levels will be required clears the way for similarly ambitious modelling projects to attempt to understand the interplay of the multitudinous factors at play in obesity, and to develop productive collaborations with population health researchers and governmental bodies.

\section{Conclusion}

While population health research has contributed to numerous high-profile health successes in modern times, there remain some highly complex, `wicked' problems which defy traditional methods of epidemiological analyses, and have resisted our attempts to develop effective interventions at the population level.  These wicked issues in population health fall into the category of post-normal science -- `These issues are urgent and of high public and political concern; the people involved hold strong positions based on their values, and the science is complex, incomplete and uncertain' \citep[][p. 163]{gluckman2014}. Whereas traditional epidemiological methods are designed to reduce uncertainty about specific causal relationships, reliance purely on these methods inhibits the ability of the population health research community to say anything about complex wicked issues that urgently require solutions and where the scientific voice is non-existent or overwhelmed by caveats and uncertainty. Methods designed to illuminate complex interrelationships are unlikely to provide certain answers, but can provide an important contribution to the debate and may also provide a mechanism to bring together a range of alternative sources of knowledge. 

As outlined above, there are numerous areas in which a complexity-inspired approach could contribute to efforts to develop more robust population health interventions for these wicked issues.  Even in these brief summaries in only three major areas of research, it is clear that developing interventions will require a broader, systems-based perspective due to the significant societal and behavioural change required to see improvement in these areas.  The complex systems science community is notable for its ability to foster computational approaches to an enormous range of domains, from artificial chemistries to simulated societies, by cohering around the mechanics and principles which can describe the complex behaviour of these seemingly disparate systems.  We suggest that this perspective makes this approach uniquely well-suited to undertake ambitious and challenging modelling projects aimed squarely at these wicked health problems.

Such efforts would require significant investment, not just in terms of time and finances, but in the form of interdisciplinary communication, which is no simple undertaking.  Developing a common language between population health and complex systems will take time, but the potential benefit to efforts to enhance the health and wellbeing of millions of people seem worth the risk of occasional frustration and misunderstanding.

\section{Acknowledgments}

M McC holds a Medical Research Council/University fellowship supported by MRC partnership grant (MC/PC/13 027). ES, LM and MMcC are part of the Complexity in Health Improvement Programme supported by the Medical Research Council (MC\_UU\_12017/14) and the Chief Scientist Office (SPHSU14).


\section*{References}

\bibliography{SSMbib}

\begin{table}
%%\hyphenpenalty10000
\centering
 \begin{tabular}{| p{2.3cm} | p{2.3cm} | p{2.3cm} | p{2.3cm} |} 
 \hline
 \textbf{Research\newline Question} & \textbf{Policy\newline Decisions} & \textbf{Research\newline Method} & \textbf{Research\newline Input} \\ [0.5ex] 
 \hline\hline
Does device X reduce health outcome Y? & Provide or prevent access to device X & Randomised trial allocating X or no X & Participant data, device utilisation \\ 
 \hline
Does device X reduce health outcome Y?\newline(Impossible to conduct RCT) & Provide or prevent access to device X & Cohort study of device X use\newline(Causal inference methods) & Participant data, device utilisation, recontact and follow-up \\
 \hline
Will people use device X? & Provide or prevent access to device X & Qualitative interviews & Participant interviews, transcription analysis and interpretation \\
 \hline
What influences (or could influence) population patterns in device X use? & Identify conditions of provision or prevention in local context & Agent-based Models & Policymaker discussion, theory, quant data, interview data \\ 
 \hline
\end{tabular}
\caption{Comparing research methods, inputs and outcomes for differing research questions}
\label{table:1}
\end{table}

\end{document}
